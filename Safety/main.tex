\documentclass[12pt]{ctexart}
\usepackage{geometry}
\geometry{a4paper,margin=2cm}
\usepackage{longtable}
\usepackage{array}

\begin{document}

\section*{太阳灶实验安全隐患清单及应对措施}

\begin{longtable}{p{2.1cm}p{3.4cm}p{4.2cm}p{5.5cm}}
\hline
\textbf{类别} & \textbf{具体隐患} & \textbf{风险说明} & \textbf{防范与应对措施} \\
\hline
\endfirsthead
\hline
\textbf{类别} & \textbf{具体隐患} & \textbf{风险说明} & \textbf{防范与应对措施} \\
\hline
\endhead
\hline
\endfoot

用电安全 &
充电宝过热、鼓包或起火 &
屋顶日照强,充电宝长时间暴晒或大电流输出,可能导致温度过高甚至热失控起火 &
选择带认证的合格充电宝;避免放在太阳下直晒,可放在阴凉处或隔热盒内;实验中定时触摸检查温度,过热立即停止使用;严禁使用外壳破损、鼓包的电池设备 \\

用电安全 &
连接线老化、破皮或被踩踏拉扯 &
屋顶设备、人员移动时容易踩到线缆,导致短路、打火或绊倒跌落 &
使用完好线缆,并用胶带或线槽固定在线路通道边缘;尽量缩短线缆长度,避免跨越通道;禁止用力拉拽接口 \\

用电安全 &
设备功率与充电宝不匹配 &
负载过大导致充电宝过流、保护或损坏 &
使用前核对实验设备电压、电流需求,确保在充电宝额定范围内;必要时分时分组使用设备,不同时大功率工作 \\

设备安全 &
太阳灶聚焦点引燃可燃物 &
聚焦光斑温度高,可点燃纸张、塑料、干叶等易燃物,引发火灾 &
聚焦区域周围清理2--3米范围内可燃物;严禁在焦点附近放置与实验无关物品;实验旁配备干粉灭火器或沙土桶,能熟练使用 \\



设备安全 &
金属部件高温烫伤 &
聚焦后锅具、支架、反射面边缘温度高,误碰易烫伤 &
显眼位置张贴“高温勿碰”提示;操作时佩戴隔热手套;搬动设备前先确认已经充分冷却 \\

建筑安全 &
屋顶承载和设备摆放不当 &
集中摆放重物或在屋顶薄弱区域活动,可能损伤屋面结构或防水层 &
了解屋顶承重限制,不在边缘或有裂缝处集中摆放重设备;设备下垫放木板或橡胶垫,避免硬物直接压在防水层上;实验结束后检查屋面是否损坏或渗漏 \\

建筑安全 &
人员靠近屋顶边缘坠落 &
观察、拍照时不慎后退或被他人碰撞,存在坠落风险 &
划定安全线,所有人员活动区域距离屋顶边缘不少于1.5--2米;严禁站上女儿墙、护栏或其他高处;安排专人负责提醒与巡视 \\

防火安全 &
屋顶可燃物堆积(杂物、干叶等) &
一旦起火,屋顶开阔且风大,火势蔓延快,难以控制 &
实验前对屋顶进行清理,移走纸箱、木板等可燃堆积物;避免在大风、空气极度干燥天气进行实验;预先确认最近的灭火器及消防栓位置 \\

防火安全 &
风力突变导致设备倾倒起火或砸人 &
强风吹倒太阳灶或支架,可能损坏设备、砸伤人员或引燃附近物品 &
选择稳固支架,并用沙袋/配重块固定;风力较大(如树枝明显摇晃)时暂停或取消实验;操作时不得离开设备无人看管 \\

人员安全 &
高温暴晒、中暑 &
屋顶温度高、暴晒时间长,学生易中暑或脱水 &
控制单次实验时长,中间安排休息;提供充足饮用水和遮阳区;出现头晕、恶心、心慌等症状时立即停止实验,转移至阴凉处,并视情况送医 \\

人员安全 &
滑倒、绊倒 &
屋顶地面有水、沙或线缆,易造成摔倒伤害 &
保持地面干燥整洁;线缆和器材摆放有序,避免横跨通道;要求穿防滑鞋,禁止追跑打闹 \\

管理措施 &
缺乏成人现场监管与应急预案 &
学生独立操作时遇突发情况应对不足 &
每次实验前向指导教师和实验室负责人报备实验时间、地点和内容;确保有教师或有经验的成年人全程在场;制作简要应急流程(起火、烫伤、坠落等)并提前向参与人员说明 \\

管理措施 &
参与人数过多、秩序混乱 &
围观人员过多靠近设备或屋顶边缘,增加事故概率 &
控制实验参与人数,分批进行;非操作人员在划定的安全线外观察 \\

\end{longtable}

\end{document}
